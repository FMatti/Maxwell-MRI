\documentclass[11pt, a4paper]{article}

\usepackage{style}

\institution{EPFL}
\project{Project CSE I}
\title{Notes}
\author{Fabio Matti}
\supervisor{Prof. Fabio Nobile \\ Dr. Davide Pradovera}
\date{\today}

\begin{document}

\maketitle

\section{Finite element method}
\label{sec:fem}

\subsection{The poisson equation}
\label{subsec:poisson}

We aim to solve an equation of the form

\begin{equation}
    - \Delta u(\mathbf{x}) = f(\mathbf{x}) \label{equ:poisson}
\end{equation}

on a domain $\mathbf{x} \in \Omega$, with a solution $u(\mathbf{x})$
that satisfies a certain boundary condition $u(\mathbf{x}) = u_d(\mathbf{x})$
for all $x \in \partial \Omega$ that lie on the border of $\Omega$.

To do this, we first convert this equation to its weak form
by multiplying both sides with a arbitrary test function
$v(\mathbf{x})$, which vanishes on the border (i.e. $v(mathbf{x}) = 0, \forall
\mathbf{x} \in \partial \Omega$), and by then integrating over all of $\Omega$:

\begin{equation}
    - \int_{\Omega} \Delta u(\mathbf{x}) v(\mathbf{x}) d\mathbf{x} = \int_{\Omega} f(\mathbf{x}) v(\mathbf{x}) d\mathbf{x} 
\end{equation}

We may now rearrange the gradient product rule $\nabla (a b) = (\nabla a) b + a (\nabla b)$ and
Gauss' theorem (as long as $v(x)$ is differentiable in a neighborhood of $\Omega$)
combined with the fact that $v(\mathbf{x})$ vanishes on $\partial
\Omega$ to convert the right-hand side to

\begin{align}
    - \int_{\Omega} \Delta u(\mathbf{x}) v(\mathbf{x}) d\mathbf{x} &= - \int_{\Omega} \nabla ( \nabla u(\mathbf{x}) v(\mathbf{x}) ) d\mathbf{x} + \int_{\Omega} \nabla u(\mathbf{x}) \nabla v(\mathbf{x}) d\mathbf{x} \notag \\ 
    &= - \int_{\partial \Omega} \nabla u(\mathbf{x}) v(\mathbf{x}) d\boldsymbol{\omega} + \int_{\Omega} \nabla u(\mathbf{x}) \nabla v(\mathbf{x}) d\mathbf{x} \notag \\ 
    &= \int_{\Omega} \nabla u(\mathbf{x}) \nabla v(\mathbf{x}) d\mathbf{x}
\end{align}

Consequently, the weak formulation of the problem is to find $u(\mathbf{x})$, such
that for arbitrary $v(\mathbf{x})$, we have

\begin{equation}
    \int_{\Omega} \nabla u(\mathbf{x}) \nabla v(\mathbf{x}) d\mathbf{x} = \int_{\Omega} f(\mathbf{x}) v(\mathbf{x}) d\mathbf{x}
\end{equation}

To simplify and generalize the notation, we may use the linear form $L:V \to \mathbb{R}$ as 

\begin{equation}
    L(v) = \int_{\Omega} f(\mathbf{x}) v(\mathbf{x}) d\mathbf{x}
\end{equation}

and also the bilinear form $a: V \times V \to \mathbb{R}$

\begin{equation}
    a(u, v) = \int_{\Omega} \nabla u(\mathbf{x}) \nabla v(\mathbf{x}) d\mathbf{x}
\end{equation}

\subsection{Example: One dimensional poisson equation}
\label{subsec:1dpoiss}

To illustrate the choice of basis functions, we will now consider the simple one 
dimensional case $\Omega = [a, b]$, such that the weak formulation of the problem
turns into 

\begin{equation}
    \int_a^b u'(x) v'(x) dx = \int_a^b f(x) v(x) dx
\end{equation}

We now subdivide the domain $[a, b]$ into $M$ subintervals, each of length
$h=(b - a)/M$, with nodes at $x_k = a + hk, k \in \{0, 1, \dots, M\}$. We
proceed to choose as the basis functions the class of the piecewise linear
Lagrange interpolating polynomials on $[x_k, x_{k+1}], k \in \{0, 1, \dots, M\}$,
defined as 

\begin{equation}
    v_k(x) = \frac{x-x_{k-1}}{x_{k}-x_{k-1}} \mathbf{1}_{\{x \in [x_{k-1}, x_k]\}} + 
            \frac{x_{k+1}-x}{x_{k+1}-x_k} \mathbf{1}_{\{x \in [x_k, x_{k+1}]\}}
\end{equation}

If we now interpolate $f(x)$ and $u(x)$ as piecewise linear Lagrange polynomaials,
we get the representation

\begin{align}
    f(x) &\approx \sum_{i=1}^{M} f(x_{i-1})\frac{x-x_{i}}{x_{i-1} - x_{i}} + f(x_i)\frac{x-x_{i-1}}{x_{i} - x_{i-1}} \notag \\
     &= \sum_{i=1}^{M-1} f(x_i)v_i(x)
\end{align}

and analogously

\begin{equation}
    u(x) = \sum_{i=1}^{M-1} u(x_i)v_i(x)
\end{equation}

We now restricted ourselves to the discrete variational formulation of the problem

\begin{equation}
    \sum_{i=1}^{M-1} u(x_i) \int_a^b v_i'(x) v_j'(x) dx = \sum_{i=1}^{M-1} f(x_i) \int_a^b v_i(x) v_j(x) dx
\end{equation}

which needs to be satisfied for all $j \in \{0, 1, \dots, M\}$.

This equation can be rewritten in terms of two matrices $\mathbf{K}$ and $\mathbf{L}$
which we define as

\begin{align}
    K_{ij} &= \int_a^b v_i(x) v_j(x) dx \\
    L_{ij} &= \int_a^b v_i'(x) v_j'(x) dx
\end{align}

such that we get

\begin{equation}
    \sum_{i=1}^{M-1} u(x_i) L_{ij} = \sum_{i=1}^{M-1} f(x_i) K_{ij}
\end{equation}

Notice, that we only need the entries $K_{ij}$ and $L_{ij}$ with
$i \in \{1, 2, \dots, M-1\}$, since we already know the boundary conditions 
of $u(x)$ at $x = x_0$ and $x = x_M$.

We realize, that the $L_2$ inner product of $v_i(x)$ with $v_j(x)$ (and
consequently also the one of $v_i'(x)$ with $v_j'(x)$) is zero for
all $|i-j| > 1$, hence, we distinguish two different cases.

\begin{enumerate}
    \item $i = j$: Here, the inner product turns out to be
    \begin{align}
        \int_a^b v_i(x) v_i(x) dx &= \int_{a}^{b} \left(\frac{x-x_{i-1}}{x_{i}-x_{i-1}} \right)^2 \mathbf{1}_{\{x \in [x_{i-1}, x_i]\}} + 
        \left(\frac{x_{i+1}-x}{x_{i+1}-x_i}\right)^2 \mathbf{1}_{\{x \in [x_i, x_{i+1}]\}} dx \notag \\ 
        &= 2 \int_{x_{i-1}}^{x_i} \left(\frac{x-x_{i-1}}{x_{i}-x_{i-1}} \right)^2 dx \notag \\ 
        &= \frac{2}{h^2} \int_{x_{i-1}-x_{i-1}}^{x_i - x_{i-1}} u^2 du \notag \\
        &= \frac{2}{h^2} \frac{1}{3} h^3 \notag \\
        &= \frac{2h}{3} \notag \\
     \end{align}
     and for the derivatives it is
     \begin{align}
        \int_a^b v_i'(x) v_i'(x) dx &= \int_{a}^{b} \left(\frac{1}{x_{i}-x_{i-1}} \right)^2 \mathbf{1}_{\{x \in [x_{i-1}, x_i]\}} + 
        \left(\frac{-1}{x_{i+1}-x_i}\right)^2 \mathbf{1}_{\{x \in [x_i, x_{i+1}]\}} dx \notag \\ 
        &= 2 \int_{x_{i-1}}^{x_i} \left(\frac{1}{x_{i}-x_{i-1}} \right)^2 dx \notag \\ 
        &= \frac{2}{h^2} \int_0^h 1 du \notag \\
        &= \frac{2}{h}
     \end{align}
    \item $|i - j| = 1$: Here, we can limit ourselves to the case where $j = i+1$,
    since the other case is fully symmetric. We calculate
    \begin{align}
        \int_a^b v_i(x) v_{i+1}(x) dx &= \int_{a}^{b} \frac{x_{i+1}-x}{x_{i+1}-x_i} \frac{x-x_i}{x_{i+1}-x_i} \mathbf{1}_{\{x \in [x_i, x_{i+1}]\}} dx \notag \\ 
            &= \int_{x_i}^{x_{i+1}} \frac{x_{i+1}-x}{x_{i+1}-x_i} \frac{x-x_i}{x_{i+1}-x_i} dx \notag \\ 
            &= \frac{1}{h^2} \int_{x_i-x_i}^{x_{i+1}-x_i} (x_{i+1}-x_i-u) u du \notag \\ 
            &= \frac{1}{h^2} \int_0^h (h - u) u du \notag \\
            &= \frac{1}{h^2} (\frac{h^3}{2} - \frac{h^3}{3}) \notag \\ 
            &= \frac{h}{6}
    \end{align}
    and for the derivative it is 
    \begin{align}
        \int_a^b v_i'(x) v_{i+1}'(x) dx &= \int_{a}^{b} \frac{-1}{x_{i+1}-x_i} \frac{1}{x_{i+1}-x_i} \mathbf{1}_{\{x \in [x_i, x_{i+1}]\}} dx \notag \\ 
            &= -\frac{1}{h^2} \int_{x_i}^{x_{i+1}} 1 dx \notag \\ 
            &= -\frac{1}{h}
    \end{align}
\end{enumerate}

Now, using the previously defined matrices $\mathbf{K}_{ij}$ and 
$\mathbf{L}_{ij}$, we get the matrix equation 

\begin{equation}
    \mathbf{L}u = \mathbf{K}f
\end{equation}

with 

\begin{align}
    u &= (u_0, u(x_1), \dots, u_M)^T \\
    f &= (f(x_0), f(x_1), \dots, f(x_{M}))^T \\
    \mathbf{L} &= \begin{pmatrix}
        1 & & & & \\
        \frac{2}{h} & -\frac{1}{h} & & & \\
        -\frac{1}{h} & \frac{2}{h} & -\frac{1}{h} & & \\
        & -\frac{1}{h} & \frac{2}{h} & \ddots & & \\
        & & -\frac{1}{h} & \ddots  & -\frac{1}{h} \\
        & & & \ddots & \frac{2}{h} \\
        & & & & 1 \\
    \end{pmatrix} \\
    \mathbf{K} &= \begin{pmatrix}
        \frac{u_0}{f(x_0)} & & & & \\
        \frac{2h}{3} & \frac{h}{6} & & & \\
        \frac{h}{6} & \frac{2h}{3} & \frac{h}{6} & & \\
        & \frac{h}{6} & \frac{2h}{3} & \ddots & & \\
        & & \frac{h}{6} & \ddots  & \frac{h}{6} \\
        & & & \ddots & \frac{2h}{3} \\
        & & & & \frac{u_M}{f(x_M)} \\
    \end{pmatrix} \\
\end{align}

Here, we have adjusted the first rows in $\mathbf{L}$ and $\mathbf{K}$, such that
the boundary conditions are necessarily satisfied.
To obtain the finite element solution, we simply solve this linear system.

\bibliography{biblio.bib}

\end{document}