\documentclass[11pt, a4paper]{article}

\usepackage{style}

\institution{EPFL}
\project{Project CSE I}
\title{Notes}
\author{Fabio Matti}
\supervisor{Prof. Fabio Nobile \\ Dr. Davide Pradovera}
\date{\today}

\begin{document}

\maketitle

\section{Finite element method}
\label{sec:fem}

\subsection{The poisson equation}
\label{subsec:poisson}

\textit{Taken from FEniCS manual (too lazy for bibtex...)}

We aim to solve an equation of the form

\begin{equation}
    - \Delta u(\mathbf{x}) = f(\mathbf{x}) \label{equ:poisson}
\end{equation}

on a domain $\mathbf{x} \in \Omega$, with a solution $u(\mathbf{x})$
that satisfies a certain boundary condition $u(\mathbf{x}) = u_d(\mathbf{x})$
for all $x \in \partial \Omega$ that lie on the border of $\Omega$.

To do this, we first convert this equation to its weak form
by multiplying both sides with a arbitrary test function
$v(\mathbf{x})$, which vanishes on the border (i.e. $v(mathbf{x}) = 0, \forall
\mathbf{x} \in \partial \Omega$), and by then integrating over all of $\Omega$:

\begin{equation}
    - \int_{\Omega} \Delta u(\mathbf{x}) v(\mathbf{x}) d\mathbf{x} = \int_{\Omega} f(\mathbf{x}) v(\mathbf{x}) d\mathbf{x} 
\end{equation}

We may now rearrange the gradient product rule $\nabla (a b) = (\nabla a) b + a (\nabla b)$ and
Gauss' theorem (as long as $v(x)$ is differentiable in a neighborhood of $\Omega$)
combined with the fact that $v(\mathbf{x})$ vanishes on $\partial
\Omega$ to convert the right-hand side to

\begin{align}
    - \int_{\Omega} \Delta u(\mathbf{x}) v(\mathbf{x}) d\mathbf{x} &= - \int_{\Omega} \nabla ( \nabla u(\mathbf{x}) v(\mathbf{x}) ) d\mathbf{x} + \int_{\Omega} \nabla u(\mathbf{x}) \nabla v(\mathbf{x}) d\mathbf{x} \notag \\ 
    &= - \int_{\partial \Omega} \nabla u(\mathbf{x}) v(\mathbf{x}) d\boldsymbol{\omega} + \int_{\Omega} \nabla u(\mathbf{x}) \nabla v(\mathbf{x}) d\mathbf{x} \notag \\ 
    &= \int_{\Omega} \nabla u(\mathbf{x}) \nabla v(\mathbf{x}) d\mathbf{x}
\end{align}

Consequently, the weak formulation of the problem is to find $u(\mathbf{x})$, such
that for arbitrary $v(\mathbf{x})$, we have

\begin{equation}
    \int_{\Omega} \nabla u(\mathbf{x}) \nabla v(\mathbf{x}) d\mathbf{x} = \int_{\Omega} f(\mathbf{x}) v(\mathbf{x}) d\mathbf{x}
\end{equation}

To simplify and generalize the notation, we may use the linear form $L:V \to \mathbb{R}$ as 

\begin{equation}
    L(v) = \int_{\Omega} f(\mathbf{x}) v(\mathbf{x}) d\mathbf{x}
\end{equation}

and also the bilinear form $a: V \times V \to \mathbb{R}$

\begin{equation}
    a(u, v) = \int_{\Omega} \nabla u(\mathbf{x}) \nabla v(\mathbf{x}) d\mathbf{x}
\end{equation}

\subsection{Example: One dimensional poisson equation}
\label{subsec:1dpoiss}

\textit{Initial idea taken from Wikipedia article about FEM.}

To illustrate the choice of basis functions, we will now consider the simple one 
dimensional case $\Omega = [a, b]$, such that the weak formulation of the problem
turns into 

\begin{equation}
    \int_a^b u'(x) v'(x) dx = \int_a^b f(x) v(x) dx
\end{equation}

We now subdivide the domain $[a, b]$ into $M$ subintervals, each of length
$h=(b - a)/M$, with nodes at $x_k = a + hk, k \in \{0, 1, \dots, M\}$. We
proceed to choose as the basis functions the class of the piecewise linear
Lagrange interpolating polynomials on $[x_k, x_{k+1}], k \in \{0, 1, \dots, M\}$,
defined as 

\begin{equation}
    v_k(x) = \frac{x-x_{k-1}}{x_{k}-x_{k-1}} \mathbf{1}_{\{x \in [x_{k-1}, x_k]\}} + 
            \frac{x_{k+1}-x}{x_{k+1}-x_k} \mathbf{1}_{\{x \in [x_k, x_{k+1}]\}}
\end{equation}

If we now interpolate $f(x)$ and $u(x)$ as piecewise linear Lagrange polynomaials,
we get the representation

\begin{align}
    f(x) &\approx \sum_{i=1}^{M} f(x_{i-1})\frac{x-x_{i}}{x_{i-1} - x_{i}} + f(x_i)\frac{x-x_{i-1}}{x_{i} - x_{i-1}} \notag \\
     &= \sum_{i=1}^{M-1} f(x_i)v_i(x)
\end{align}

and analogously

\begin{equation}
    u(x) = \sum_{i=1}^{M-1} u(x_i)v_i(x)
\end{equation}

We now restricted ourselves to the discrete variational formulation of the problem

\begin{equation}
    \sum_{i=1}^{M-1} u(x_i) \int_a^b v_i'(x) v_j'(x) dx = \sum_{i=1}^{M-1} f(x_i) \int_a^b v_i(x) v_j(x) dx
\end{equation}

which needs to be satisfied for all $j \in \{0, 1, \dots, M\}$.

This equation can be rewritten in terms of two matrices $\mathbf{K}$ and $\mathbf{L}$
which we define as

\begin{align}
    K_{ij} &= \int_a^b v_i(x) v_j(x) dx \\
    L_{ij} &= \int_a^b v_i'(x) v_j'(x) dx
\end{align}

such that we get

\begin{equation}
    \sum_{i=1}^{M-1} u(x_i) L_{ij} = \sum_{i=1}^{M-1} f(x_i) K_{ij}
\end{equation}

Notice, that we only need the entries $K_{ij}$ and $L_{ij}$ with
$i \in \{1, 2, \dots, M-1\}$, since we already know the boundary conditions 
of $u(x)$ at $x = x_0$ and $x = x_M$.

We realize, that the $L_2$ inner product of $v_i(x)$ with $v_j(x)$ (and
consequently also the one of $v_i'(x)$ with $v_j'(x)$) is zero for
all $|i-j| > 1$, hence, we distinguish two different cases.

\begin{enumerate}
    \item $i = j$: Here, the inner product turns out to be
    \begin{align}
        \int_a^b v_i(x) v_i(x) dx &= \int_{a}^{b} \left(\frac{x-x_{i-1}}{x_{i}-x_{i-1}} \right)^2 \mathbf{1}_{\{x \in [x_{i-1}, x_i]\}} + 
        \left(\frac{x_{i+1}-x}{x_{i+1}-x_i}\right)^2 \mathbf{1}_{\{x \in [x_i, x_{i+1}]\}} dx \notag \\ 
        &= 2 \int_{x_{i-1}}^{x_i} \left(\frac{x-x_{i-1}}{x_{i}-x_{i-1}} \right)^2 dx \notag \\ 
        &= \frac{2}{h^2} \int_{x_{i-1}-x_{i-1}}^{x_i - x_{i-1}} u^2 du \notag \\
        &= \frac{2}{h^2} \frac{1}{3} h^3 \notag \\
        &= \frac{2h}{3} \notag \\
     \end{align}
     and for the derivatives it is
     \begin{align}
        \int_a^b v_i'(x) v_i'(x) dx &= \int_{a}^{b} \left(\frac{1}{x_{i}-x_{i-1}} \right)^2 \mathbf{1}_{\{x \in [x_{i-1}, x_i]\}} + 
        \left(\frac{-1}{x_{i+1}-x_i}\right)^2 \mathbf{1}_{\{x \in [x_i, x_{i+1}]\}} dx \notag \\ 
        &= 2 \int_{x_{i-1}}^{x_i} \left(\frac{1}{x_{i}-x_{i-1}} \right)^2 dx \notag \\ 
        &= \frac{2}{h^2} \int_0^h 1 du \notag \\
        &= \frac{2}{h}
     \end{align}
    \item $|i - j| = 1$: Here, we can limit ourselves to the case where $j = i+1$,
    since the other case is fully symmetric. We calculate
    \begin{align}
        \int_a^b v_i(x) v_{i+1}(x) dx &= \int_{a}^{b} \frac{x_{i+1}-x}{x_{i+1}-x_i} \frac{x-x_i}{x_{i+1}-x_i} \mathbf{1}_{\{x \in [x_i, x_{i+1}]\}} dx \notag \\ 
            &= \int_{x_i}^{x_{i+1}} \frac{x_{i+1}-x}{x_{i+1}-x_i} \frac{x-x_i}{x_{i+1}-x_i} dx \notag \\ 
            &= \frac{1}{h^2} \int_{x_i-x_i}^{x_{i+1}-x_i} (x_{i+1}-x_i-u) u du \notag \\ 
            &= \frac{1}{h^2} \int_0^h (h - u) u du \notag \\
            &= \frac{1}{h^2} (\frac{h^3}{2} - \frac{h^3}{3}) \notag \\ 
            &= \frac{h}{6}
    \end{align}
    and for the derivative it is 
    \begin{align}
        \int_a^b v_i'(x) v_{i+1}'(x) dx &= \int_{a}^{b} \frac{-1}{x_{i+1}-x_i} \frac{1}{x_{i+1}-x_i} \mathbf{1}_{\{x \in [x_i, x_{i+1}]\}} dx \notag \\ 
            &= -\frac{1}{h^2} \int_{x_i}^{x_{i+1}} 1 dx \notag \\ 
            &= -\frac{1}{h}
    \end{align}
\end{enumerate}

Now, using the previously defined matrices $\mathbf{K}_{ij}$ and 
$\mathbf{L}_{ij}$, we get the matrix equation 

\begin{equation}
    \mathbf{L}u = \mathbf{K}f
\end{equation}

with 

\begin{align}
    u &= (u_0, u(x_1), \dots, u_M)^T \\
    f &= (f(x_0), f(x_1), \dots, f(x_{M}))^T \\
    \mathbf{L} &= \begin{pmatrix}
        1 & & & & \\
        \frac{2}{h} & -\frac{1}{h} & & & \\
        -\frac{1}{h} & \frac{2}{h} & -\frac{1}{h} & & \\
        & -\frac{1}{h} & \frac{2}{h} & \ddots & & \\
        & & -\frac{1}{h} & \ddots  & -\frac{1}{h} \\
        & & & \ddots & \frac{2}{h} \\
        & & & & 1 \\
    \end{pmatrix} \\
    \mathbf{K} &= \begin{pmatrix}
        \frac{u_0}{f(x_0)} & & & & \\
        \frac{2h}{3} & \frac{h}{6} & & & \\
        \frac{h}{6} & \frac{2h}{3} & \frac{h}{6} & & \\
        & \frac{h}{6} & \frac{2h}{3} & \ddots & & \\
        & & \frac{h}{6} & \ddots  & \frac{h}{6} \\
        & & & \ddots & \frac{2h}{3} \\
        & & & & \frac{u_M}{f(x_M)} \\
    \end{pmatrix} \\
\end{align}

Here, we have adjusted the first rows in $\mathbf{L}$ and $\mathbf{K}$, such that
the boundary conditions are necessarily satisfied.
To obtain the finite element solution, we simply solve this linear system.

\section{Maxwell's equations}
\label{sec:maxwell}

Let $\mathbf{E} = (E_1, E_2, E_3)^T$ denote the electric field,
$\mathbf{B} = (B_1, B_2, B_3)^T$ the magnetic field strength, and 
$\mathbf{j} = (j_1, j_2, j_3)^T$ the electric current density. 
We suppose Maxwell's equations hold:

\begin{align}
    \nabla \cdot (\epsilon \mathbf{E}) &= \rho \\
    \nabla \cdot \mathbf{B} &= 0 \\
    \nabla \times \mathbf{E} &= -\partial_t \mathbf{B} \\
    \nabla \times (\mu^{-1} \mathbf{B}) &= \partial_t (\epsilon \mathbf{E}) + \mathbf{j} \label{equ:mw4}
\end{align}

We can therefore write $\mathbf{B} = \nabla \times \mathbf{A}$ for some vector
potential $\mathbf{A}$, and $\mathbf{E} = -\nabla \phi - \partial_t \mathbf{A}$ for some
scalar potential $\phi$. Plugging these identities into (\ref{equ:mw4}), we get 

\begin{equation}
    \nabla \times (\mu^{-1} \nabla \times \mathbf{A}) = \partial_t \nabla \phi -
    \partial_t^2 \mathbf{A} + \mathbf{j} \label{equ:mwvecpot}
\end{equation}

We may choose $\nabla \phi = 0$ as a gauge, and introduce a harmonic time dependence
of $\mathbf{A}$ and $\mathbf{j}$ with frequencies $\omega$, such that $\mathbf{A}(\mathbf{x}, t) = 
\mathbf{A}(\mathbf{x})\exp(i \omega t)$ and $\mathbf{j}(\mathbf{x}, t) = 
\mathbf{j}(\mathbf{x})\exp(i \omega t)$. Plugging this into (\ref{equ:mwvecpot})
yields us

\begin{equation}
    \nabla \times (\mu^{-1} \nabla \times \mathbf{A}) - \omega^2 \mathbf{A} = \mathbf{j} \label{equ:mwtimeharm}
\end{equation}

We reduce this equation to its weak formulation, by multiplying it with a vector-valued 
function $\mathbf{v} \in H_{\text{curl}}(\Omega)$, where we denoted

\begin{equation}
    H_{\text{curl}}(\Omega) = \{u : \Omega \to \mathbb{C}, ~\text{such that}~ u \in L^2(\mathbb{C})^3,
    \nabla \times u \in L^2(\mathbb{C})^3 \}
\end{equation}

and by integrating over all of $\Omega$:

\begin{equation}
    \int_{\Omega} (\nabla \times ({\mu^{-1} \nabla \times \mathbf{A}})) \cdot \mathbf{v}
    - \omega^2 \int_{\Omega} \mathbf{A} \cdot \mathbf{v} = \int_{\Omega} \mathbf{j} \cdot \mathbf{v} \label{equ:mwweak}
\end{equation}

To further simplify this expression, we will derive an identity for the scalar product
of a vector-valued function $\mathbf{v}$ with the curl of a vector-valued function 
$\mathbf{u}$. For this, we use the completely antisymmetric tensor $\varepsilon_{ijk}$
(frequently referred to as the Levi-Civita tensor), to rewrite the $k$-th component
of the curl as

\begin{equation}
    (\nabla \times \mathbf{u})_k = \sum_i \sum_j \varepsilon_{ijk} \partial_i u_j
\end{equation}

where $\partial_i$ denotes the partial derivative with respect to the $i$-th coordinate
direction. Rewriting the scalar product as a sum and identifying $\mathbf{u} = \mu^{-1}
\nabla \times \mathbf{A}$, we apply the product rule to get

\begin{align}
    (\nabla \times \mathbf{u}) \cdot \mathbf{v} &= \sum_k (\nabla \times \mathbf{u})_k v_k \notag \\ 
    &= \sum_k (\sum_i \sum_j \varepsilon_{ijk} \partial_i u_j) v_k \notag \\ 
    &= \sum_k \sum_i \sum_j \partial_i (\varepsilon_{ijk} u_j v_k) - \sum_k \sum_i \sum_j u_j (\varepsilon_{ijk} \partial_i v_k) \notag \\ 
    &= \sum_k \sum_i \sum_j \partial_i (\varepsilon_{jki} u_j v_k) - \sum_k \sum_i \sum_j u_j ((-\varepsilon_{ikj}) \partial_i v_k) \notag \\ 
    &= \sum_i \partial_i (\mathbf{u} \times \mathbf{v})_i + \sum_j u_j (\nabla \times \mathbf{v})_j \notag \\ 
    &= \nabla \cdot (\mathbf{u} \times \mathbf{v}) + \mathbf{u} \cdot (\nabla \times \mathbf{v}) \label{equ:curlidentity} 
\end{align}

Consequently, we may rewrite the double curl term in the weak formulation as 

\begin{align}
    \int_{\Omega} (\nabla \times ({\mu^{-1} \nabla \times \mathbf{A}})) \cdot \mathbf{v} &=
    \int_{\Omega} \nabla \cdot (({\mu^{-1} \nabla \times \mathbf{A}}) \times \mathbf{v})
    + \int_{\Omega} ({\mu^{-1} \nabla \times \mathbf{A}}) \cdot (\nabla \times \mathbf{v}) \notag \\
    &= \int_{\partial \Omega} (({\mu^{-1} \nabla \times \mathbf{A}}) \times \mathbf{v}) \cdot \mathbf{n}
    + \int_{\Omega} ({\mu^{-1} \nabla \times \mathbf{A}}) \cdot (\nabla \times \mathbf{v}) \notag \\
\end{align}

We will now have a look at what conditions $\mathbf{v}$ needs to satisfy, such that
the boundary term (first integral) vanishes, and we would end up with

\begin{equation}
    \int_{\Omega} ({\mu^{-1} \nabla \times \mathbf{A}}) \cdot (\nabla \times \mathbf{v})
    - \omega^2 \int_{\Omega} \mathbf{A} \cdot \mathbf{v} 
    = \int_{\Omega} \mathbf{j} \cdot \mathbf{v} \label{equ:mwweaksimple}
\end{equation}

Let $\mathbf{n}$ denote the normal
vector to $\partial \Omega$ at a point $\mathbf{x} \in \partial \Omega$.
For the boundary term to vanish, we require 

\begin{equation}
    (({\mu^{-1} \nabla \times \mathbf{A}}) \times \mathbf{v}) \cdot \mathbf{n} = 0
\end{equation}

for all $\mathbf{x} \in \partial \Omega$. Denoting $\mathbf{u} = \mu^{-1} \nabla \times \mathbf{A}$,
we rearrange 

\begin{align}
    (\mathbf{u} \times \mathbf{v}) \cdot \mathbf{n} &= \sum_k (\sum_i \sum_j \varepsilon_{ijk} u_i v_j) n_k \notag \\
     &= \sum_i u_i (\sum_j \sum_k \varepsilon_{jki} v_j n_k) \notag \\ 
     &= \mathbf{u} \cdot (\mathbf{v} \times \mathbf{n}) \label{equ:label}
\end{align}

For non-trivial $\mathbf{u}$ and $\mathbf{v}$, this expression is zero if and only if 
$\mathbf{v} \perp \mathbf{n}$, meaning $\mathbf{v}$ is orthogonal to $\partial \Omega$ 
for all $\mathbf{x} \in \partial \Omega$. 

\section{Weak derivative}
\label{sec:weak}

\textit{Taken from Quarteroni: Introduction to Finite Elements Method}

Let $\Omega \subset \mathbb{R}^d$ open. The support of $f:\Omega \to \mathbb{R}$
is defined as

\begin{equation}
    \text{supp}(f) = \overline{\{x \in \Omega ~|~ f(x) \neq 0\}}
\end{equation}

$f$ has compact support, if there exists a compact subset $K \subset \Omega$,
such that supp($f$) $\subset K$, and define

\begin{equation}
    \mathcal{D}(\Omega) = \{f \in C^{\infty}(\Omega) ~|~ f ~\text{has compact support}\}
\end{equation}

(If I remember correctly, extending this notion to $f \in C^1(\Omega)$ should yield
an almost identical treatment, unless we also include higher order (weak) partial derivatives).
Let $T: \mathcal{D} \to \mathbb{R}$, $\varphi \mapsto \langle T, \varphi \rangle = T(\varphi)$
be a linear map. We say that $T$ is continuous, if

\begin{equation}
    \lim_{n\to\infty} \langle T, \varphi_n \rangle = \langle T, \varphi \rangle
\end{equation}

with $\{\varphi_k\}_{k \in \mathbb{N}} \subset \mathcal{D}(\Omega)$ converging to $\varphi$.
Such (linear and continuous) maps are called distribution on $\mathcal{D}(\Omega)$,
and they form the space of distributions $\mathcal{D}'(\Omega)$.

The (weak) partial coordinate-derivatives of $T$ (namely $\partial_i T, ~i \in \{1, \dots, d\}$)
are characterized by distributions that satisfy

\begin{equation}
    \langle \partial_i T, \varphi \rangle = - \langle T, \partial_i \varphi \rangle 
\end{equation}

for all $\varphi \in \mathcal{D}(\Omega)$. 

Interesting for us is mainly the following case:
Given a function $f \in L^2(\Omega)$, we define a distribution $T_f \in \mathcal{D}'(\Omega)$
to be 

\begin{equation}
    \langle T_f , \varphi \rangle = \int_{\Omega} f(x) \varphi(x) dx
\end{equation}

for all $\varphi \in \mathcal{D}(\Omega)$.

This allows us to define a weak derivative to functions that are (in the classical
sense) not differentiable (i.e. not in $C^1(\Omega)$). Consider for example the 
absolute value function $|\cdot| \in L_2(K)$ where $K \subset \mathbb{R}$ is compact.
Since

\begin{align}
    \int_K (\partial_x |x|) \varphi(x) dx &= -\int_K |x| \varphi'(x) dx \notag \\ 
    &= - \int_{K \cap \mathbb{R}_+} x \varphi'(x) dx - \int_{K \cap \mathbb{R}_-} (-x) \varphi'(x) dx \notag \\ 
    &= \int_{K \cap \mathbb{R}_+} \varphi(x) dx + \int_{K \cap \mathbb{R}_-} (-1) \varphi(x) dx \notag \\
    &= \int_K \text{sign}(x) \varphi(x) dx
\end{align}

we may conclude that the weak derivative of the absolute value function is therefore
the signum function. Notice, how the derivative of the absolute value function
is only not well-defined at $x=0$, i.e. on a set of zero measure. This nuisance
is circumvented when talking about the weak derivative, since the measure zero
sets have zero integral.

\section{Ideas}
\label{sec:ideas}

What might be really interesting is to instead look at the problem in space-time 
using the Maxwell tensor

\begin{equation}
    \mathbb{F} = \begin{bmatrix}
        0 & -E_1/c & -E_2/c & -E_3/c \\
        E_1/c & 0 & B_3 & -B_2 \\ 
        E_2/c & -B_3 & 0 & B_1 \\ 
        E_3/c & B_2 & -B_1 & 0
    \end{bmatrix}
\end{equation}

In the covariant formulation of the Maxwell theory, the inhomogeneous Maxwell
equations reduce to a single equation

\begin{equation}
    \partial_a F^{ab} = - J^b
\end{equation}

with the four current density $\mathbf{J} = (\mu c \rho, \mu \mathbf{j})$. The 
weak formulation of the problem could then be stated as (using Einstein's sum
convention, i.e. summing over repeated indices)

\begin{equation}
    \int_{\Omega \times \mathbb{R}} F^{ab} \partial_a v_b = \int_{\Omega \times \mathbb{R}} J^b v_b
\end{equation}

where boundary conditions are yet to be determined. If we somehow would manage
to find a suitable function space for the four-dimensional $\mathbf{v}$, it might 
be possible to find both $\mathbf{E}$ and $\mathbf{B}$ from a finite element method.

\bibliography{biblio.bib}

\end{document}